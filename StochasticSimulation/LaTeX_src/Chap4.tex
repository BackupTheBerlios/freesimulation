\chapter{Markov processes and master equations}
This chapter is devoted
to the introduction of some mathematical concepts, which allow the
correct treatment of time--dependent probabilistic phenomena. 
Such processes occur in many branches of physics. A typical example,
is for instance, the dynamics of the velocity field in a turbulent fluid.
We will introduce stochastic processes as time dependent stochastic 
variables and we will 
learn how the dynamics of a particular class of stochastic 
processes, the so--called Markov processes  is described 
with the help of differential Chapman--Kolmogorov equations. 
These concepts will be applied
in the next chapters to typical examples from statistical physics.

\section{Stochastic processes}
We have already learned in Chap. 2 that once a stochastic variable 
$X$ has been defined it is possible to define other stochastic 
variables, say $Y$, as functions of $X$ by some mapping $f$. In 
particular, the quantity $Y$ may be a function of an additional
time variable $t$, i.e.,
\begin{equation*}
Y(t) = f(X,t).
\end{equation*}
Sloppy speaking, such a quantity $Y(t)$ is called a {\em stochastic
processes}. If we insert for $X$ one of its possible values $x$
we obtain an ordinary function
\begin{equation*}
y(t) = f(x,t),
\end{equation*}
which is a {\em realization} of the stochastic process
\cite{VAN_KAMPEN}. It is 
customary in statistical physics to regard the stochastic process 
as an {\em ensemble} of such realizations.

It follows immediately from the random variable transformation 
theorem that the probability density for $Y(t)$ to take the value
$y$ at time $t$ is given by
\begin{equation*}
P_1(y,t) = \int dx  \delta(y-f(x,t)) P(x)
\end{equation*}
and, accordingly, the joint probability density that $Y$ has the
value $y_1$ at time $t_1$, the value $y_2$ at time $t_2$, 
$\ldots$, and the value $y_n$ at time $t_n$ is given by
\begin{eqnarray*}
\lefteqn{P_n(y_1,t_1;y_2,t_2; \ldots, y_n,t_n)} \\
&& = \int dx \delta(y_1-f(x,t_1))\delta(y_2-f(x,t_2)) \cdots 
   \delta(y_n-f(x,t_n))  P(x).
\end{eqnarray*}
In such a way an infinite hierarchy of joint probability densities
$P_n$ $(n=1,2,\ldots)$ is defined, which allows the evaluation of 
expectation values like
\begin{eqnarray*}
\lefteqn{<Y(t_1) Y(t_2) \cdots Y(t_n)>} \\
&& = \int dy_1 \int dy_2 \cdots \int dy_n 
        y_1 y_2 \cdots y_n P_n(y_1,t_1;y_2,t_2; \ldots, y_n,t_n).
\end{eqnarray*}
It has been shown by Kolmogorov (\cite{VAN_KAMPEN}) that the 
hierarchy of joint probability densities introduced above 
completely specifies
a stochastic process if the following four consistency conditions 
are satisfied \\
(i) $P_n \ge 0$; \\
(ii) $P_n$ is a symmetric function of the pairs $(y_1,t_1)$, 
$\ldots$, $(y_n,t_n)$; \\
(iii) $\int dy_n P_n(y_1,t_1; \ldots , y_n,t_n) =
     P_{n-1}(y_1,t_1; \ldots ; y_{n-1},t_{n-1})$; \\
(iv) $\int dy_1 P(y_1,t_1) =1$.

Thus, the hierarchy of joint probability densities constitutes an 
alternative way to define stochastic processes. 
With increasing $n$ the description of the stochastic process gets
more precise. It is important to 
make the following remarks.  The condition (iii) implies that each 
density $P_n$ includes the knowledge of all previous densities
$P_k$ with $k<n$. Furthermore, the density $P_n$ does have the 
following property if two time arguments are identical
\begin{equation*}
P_n(x,t;y_1,t;y_2,t_2; \ldots ; y_{n-1},t_{n-1}) = 
P_{n-1}(x,t;y_2,t_2; \ldots ; y_{n-1},t_{n-1})
\delta(x-y_1).
\end{equation*}
The hierarchy of probability densities is also the starting point 
for the classification of stochastic processes. A stochastic 
process is said to be purely random if events at different times 
are not correlated. In this case the joint probability density 
factorizes, i.e. we have
\begin{eqnarray*}
P_2(y_1,t_1;y_2,t_2) &=& P_1(y_1,t_1) P_1(y_2,t_2) \\
P_3(y_1,t_1;y_2,t_2;y_3,t_3) &=& P_1(y_1,t_1) P_1(y_2,t_2)P_1(y_3,t_3) 
\\
\text{and so on.} &&
\end{eqnarray*}
This means that the value of $Y$ at time $t$ is completely 
independent of its values in the past and in the future. An even
more special case occurs when the $P_1(y_i,t_i)$ are independent 
of $t_i$. In this case the same probability law governs the 
process for all times. Such processes are called
{\em Bernoulli trials} (\cite{GARDINER}). In the next section we 
will introduce the next most simple class, the {\em Markov processes},
in which the knowledge of only the presents determines the future.


\section{Markov processes}
In order to define the class of stochastic processes, which will be 
of central importance in the forthcoming theoretical discussions 
and in the examples of the next chapters, we will formulate
the {\em Markov assumption}. This assumption is formulated in 
terms of conditional probability densities which we 
will denote by
$T_n(x,t|y_1,t_1;y_2,t_2; \ldots, y_n,t_n)$. This quantity gives 
the probability that the stochastic process takes the value $x$ at 
time $t$ given that it had the value $y_1$ at time $t_1$, $y_2$
at time $t_2$, $\ldots$, $y_n$ at time $t_n$, where we assume that
$t_1< \cdots <t_n < t$.  The conditional probability density has 
the following properties \\
(i) $T_n \ge 0$, \\
(ii) $\int dx T_n = 1$,  \\
(iii) $T_n(x,t|y_1,t;y_2,t_2; \ldots, y_n,t_n) = \delta(x-y_1)$.\\
As we already know the joint probability density $P_n$ can be 
expressed with the help of Bayes' theorem through the conditional 
probability density $T_{n-1}$ as
\begin{eqnarray*}
\lefteqn{P_n(x,t;y_1,t_1;y_2,t_2; \ldots, y_{n-1},t_{n-1})} \\
&& = P_{n-1}(y_1,t_1;y_2,t_2; \ldots, y_{n-1},t_{n-1})
T_{n-1}(x,t|y_1,t_1;y_2,t_2; \ldots, y_{n-1},t_{n-1}).
\end{eqnarray*}
Now we are in the position to define the class of Markov 
processes. Let $t_1< \cdots <t_n < t_{n+1}$ be an ordered sequence 
of times. A Markov process is defined through the following 
condition for the conditional probability density of the
stochastic process
\begin{equation*}
T_{n}(y_{n+1},t_{n+1}|y_1,t;y_2,t_2; \ldots, y_{n},t_{n})
= T_{1}(y_{n+1},t_{n+1}|y_{n},t_{n}).
\end{equation*}
In other words, the conditional probability density at $t_{n+1}$
given the value of $y_n$ at time $t_n$ is uniquely determined and 
is not affected by any value of $y$ at earlier times. Thus, the
conditional probability density is determined completely by the 
knowledge of the most recent condition.
The above definition implies that for a Markov process all
$T_n$ with $n \ge 1$ can be determined from the
conditional probability density $T_1$, which will also be called
the one step 
transition probability density. As an immediate consequence a Markov 
processes is completely characterized by the knowledge of the
one step transition probability and by the probability density
$P_1$. With the help of these two functions we can reconstruct the 
whole hierarchy of probability densities. For example, we
have
\begin{eqnarray}
\label{P3}
P_3(y_1,t_1;y_2,t_2;y_3,t_3) &=& 
      P_2(y_1,t_1;y_2,t_2) T_2(y_3,t_3|y_1,t_1;y_2,t_2)   
         \nonumber \\
    & = & T_1(y_3,t_3|y_2,t_2) T_1(y_2,t_2|y_1,t_1)
           P_1(y_1,t_1).
\end{eqnarray}
Integrating the above equation (\ref{P3}) over $y_2$ we obtain 
\begin{equation}
P_2(y_1,t_1;y_3,t_3) = P_1(y_1,t_1) 
       \int dy_2 T_1(y_3,t_3|y_2,t_2) T_1(y_2,t_2|y_1,t_1).
\end{equation}
Dividing both sides by $P_1(y_1,t_1)$ we obtain an identity which 
must be obeyed by the transition probability of any Markov process
\begin{equation}
\label{CHAPMAN_KOLMOGOROV}
T_1(y_3,t_3|y_1,t_1) =  
       \int dy_2 T_1(y_3,t_3|y_2,t_2) T_1(y_2,t_2|y_1,t_1).
\end{equation}
The above identity is called the {\em Chapman--Komogorov 
equation}. It has a simple interpretation. The transition 
probability between two states $y_1$ and $y_3$ with $t_1 <t_3$
corresponds to the product of the transition probability between 
the initial state and some intermediate state and the transition 
between this intermediate state and the final state integrated
over all intermediate states.

As we already noted the functions $P_1$ and $T_1$ uniquely define 
a Markov process. However, these two functions are not arbitrary.
They must satisfy the Chapman--Kolmogorov equation and the obvious
consistency condition
\begin{equation}
\label{CONSISTENCY}
P_1(y_2,t_2) =  
       \int dy_1  T_1(y_2,t_2|y_1,t_1) P_1(y_1,t_1).
\end{equation}
For the sake of a compact notation we will write now $P=P_1$ and 
$T=T_1$.


\section{The differential Chapman--Kolmogorov equation}
We now derive a differential  form of the Chapman--Kolmogorov 
equation which is more practical for physical applications.
We will proceed in two steps. First, we introduce the concept
of generator of a stochastic process. Second, we will construct
with the help of the generator an equation of motion for the
transition probabilty density.

\subsection{The generator of a Markov process}
We consider the time evolution of the expectation value
of  a function $f(y)$.
Thus,
\begin{eqnarray*}
\frac{\partial}{\partial t} \langle f(y) \rangle &=&
 \frac{\partial}{\partial t} 
    \left\{  \int dy f(y) P(y,t)  \right\}  \\
 &= & \lim_{\Delta t \rightarrow 0} \frac{1}{\Delta t}
       \left\{  \int dy f(y) \left[P(y,t+\Delta t) - P(y,t) \right]  
       \right\}.
\end{eqnarray*}
Making use of the consistency condition (\ref{CONSISTENCY}) 
in the first term 
on the right--hand side of the above equation we obtain
\begin{equation*}
\frac{\partial}{\partial t} \langle f(y) \rangle =
   \lim_{\Delta t \rightarrow 0} \frac{1}{\Delta t}
    \left\{  \int dy \int dy' f(y) T(y,t+\Delta t|y',t) P(y',t) 
         - \int dy f(y) P(y,t) 
       \right\}.
\end{equation*}
We rename the integration variables in the positive 
term of the right--hand side of the above equation ($y \rightarrow y'$,
$y' \rightarrow y$) to obtain
\begin{equation*}
\frac{\partial}{\partial t} \langle f(y) \rangle =
   \lim_{\Delta t \rightarrow 0} \frac{1}{\Delta t}
    \left\{  \int dy  \int dy' \left[ f(y') T(y',t+\Delta t|y,t) 
         - \delta(y-y') f(y') \right] P(y,t) 
       \right\},
\end{equation*}
which we can also write as
\begin{equation*}
\frac{\partial}{\partial t} \langle f(y) \rangle =
\int dy P(y,t) \lim_{\Delta t \rightarrow 0} \frac{1}{\Delta t}
\left\{ \int dy' \left[ f(y') T(y',t+\Delta t|y,t) 
         - \delta(y-y') f(y') \right].
\right\}
\end{equation*}
At this point it is convenient to introduce the infinitesimal 
generator of a Markov process ${\cal{A}}$ as
\begin{equation}
\label{DEF_GENERATOR}
{\cal{A}}(t) f(x) = \lim_{\Delta t \rightarrow 0}
    \frac{1}{\Delta t} 
    \left[\int dy f(y)T(y,t+\Delta t|x,t) - f(x)  \right].
\end{equation}
$f(y)$ is some measurable function for which the above 
limit exists. Evidently ${\cal{A}}$ is a linear operator, which 
can be determined from the transition probability density. 
When the operator ${\cal{A}}$ operates on $f$ it describes the
change of the expectation value of $f$ in an infinitesimal time 
step. As a consequence of the Chapman--Kolmogorov equation each time 
step $t-t_1$ can be decomposed into a sequence of smaller time 
steps. So it is plausible to characterize the Markov
process by regarding infinitesimal time steps. 
The importance of the generator ${\cal{A}}$ lies in the fact
that together with some initial condition $P(x,t=0)$ it specifies
uniquely the Markov process.
The time evolution equation for the expectation value can be written 
in the compact and suggestive form
\begin{equation}
\label{EQ_EXPECTATION}
\frac{\partial}{\partial t} \langle f \rangle = 
\langle {\cal{A}} f \rangle.
\end{equation}


\subsection{The differential Chapman--Kolmogorov equation}
With the help of the generator we derive an equation of motion for
the transition probability $T$.
Multiplying equation (\ref{DEF_GENERATOR}) 
with $T(x,t|x',t')$ ($t'<t$) and 
integrating over $x$ we obtain
\begin{eqnarray*}
\lefteqn{\int dx \left[ {\cal{A}}(t) f(x) \right] T(x,t|x',t')} \\
&& = \lim_{\Delta t \rightarrow 0}
    \frac{1}{\Delta t} 
    \left[\int dy f(y)T(y,t+\Delta t|x',t') - 
        \int dx f(x) T(x,t|x',t')  \right],
\end{eqnarray*}
where we made use of the Chapman--Kolmogorov equation.
We now rename the variable $y$ on the right--hand side of the above 
equation and call it $x$ and perform the limit $\Delta t \rightarrow 
0$
\begin{equation}
\label{KOL_FORWARD_0}
\int dx \left[ {\cal{A}}(t) f(x) \right] T(x,t|x',t') =
  \int dx f(x) \frac{\partial}{\partial t} T(x,t|x',t').
\end{equation}
It is convenient to introduce the adjoint operator ${\cal{A}}^{\dagger}$
to the generator ${\cal{A}}$ according to the following definition
\begin{equation}
\label{ADJOINT}
\int dx \left[ {\cal{A}}(t) f(x) \right] T(x,t|x',t') \equiv
\int dx f(x) \left[ {\cal{A}}^{\dagger}(t)  T(x,t|x',t') \right].
\end{equation}
We will see in the next sections how the adjoint operator is
explicitely constructed.
Inserting Eq. (\ref{ADJOINT}) into Eq. (\ref{KOL_FORWARD_0})
and considering that (\ref{ADJOINT}) holds for any 
function $f(x)$ we conclude that the equation of motion for the 
transition probability is given by
\begin{equation}
\label{KOL_FORWARD}
\frac{\partial}{\partial t} T(x,t|x',t') =
{\cal{A}}^{\dagger}(t) T(x,t|x',t').
\end{equation}
We will call the above equation the {\em differential 
Chapman--Kolmogorov equation}. The differential 
Chapman--Kolmogorov equation is the central equation of this 
chapter. Together with some initial probability distribution it 
defines completely a Markov process. With its help it is possible
to compute time--dependent expectation values and multi--time 
correlation functions. Because of its importance, it is sometimes
named the {\em master equation} in the physical literature.

\section{The Liouville equation}
Let us consider a physical system whose dynamics is described
by a system of ordinary differential equations of first order
\begin{equation}
\label{ORDINARY_DIFF}
\frac{d}{dt} x(t) = g(x(t)),
\end{equation}
where $g$ is a function $R^d \rightarrow R^d$. 
It is clear that Hamiltonian systems belong to this class 
(\cite{ARNOLD}). The initial 
condition is
\begin{equation*}
x(0) = x \in R^d.
\end{equation*}
We denote the unique solution of this equation by $\phi(t,x)$,
where the $x$ stresses the dependence on the initial condition.

If $f:R^d \rightarrow R^d$ is a continuous differentiable function
then it follows from Eq. (\ref{ORDINARY_DIFF}) 
\begin{equation}\label{COORDINATE_FREE}
\frac{d}{dt} f(x(t)) = \sum_i \frac{\partial f}{\partial x_i}
          (x(t)) g^i(x(t)),
\end{equation}
where $g^i$ denotes the $i$--th component of $g$. 


With the help of these formal preliminaries it is easy to 
construct the generator of a deterministic Markov process.
Obviously we have for the expectation value
\begin{equation*}
Ef(x(t)) = f(\phi(x,t)),
\end{equation*}
where the symbol $E$ denotes the expectation value.

Inserting the above expectation value into the definition of a 
generator (\ref{DEF_GENERATOR}) we immediately obtain
\begin{eqnarray*}
{\cal{A}}_L(t) f & = & \lim_{t\rightarrow 0} \frac{1}{t}
                        \left[ Ef(x(t)) - f(x)\right] \\
             & = & \lim_{t\rightarrow 0} \frac{1}{t}
                        \left[ f(\phi(x,t)) - f(\phi(0,x))\right] \\
              & = & \frac{d}{dt} f(x) ,
\end{eqnarray*}
and finally, using Eq. (\ref{COORDINATE_FREE}),
\begin{equation*}
{\cal{A}}_L(t) f = \sum_i \frac{\partial f}{\partial x_i}(x) g^i(x).
\end{equation*}


Having determined the generator it is now straightforward to 
evaluate the corresponding differential Chapman--Kolmogorov 
equation. To this end we only 
have to determine the operator which is adjoint to
${\cal{A}}_L$ by partial integration. It is evident that we have
\begin{eqnarray}
\label{GEN_DET0}
\int dx \left[{\cal{A}}_L f(x) \right]h(x) &=&
          \sum_i \int dx \left[g^i 
           \frac{\partial f}{\partial x_i} \right] h(x) \nonumber \\
     & = & -\sum_i f \frac{\partial}{\partial x_i} g^i h(x).      
\end{eqnarray}
Since Eq. (\ref{GEN_DET0}) holds for any function $h(x)$
\begin{equation}
\label{GEN_DET_A}
{\cal{A}}_L^{\dagger} h(x) = -\sum_i \frac{\partial}{\partial x_i}
    \left(g_i h(x) \right).
\end{equation}
Inserting (\ref{GEN_DET_A}) into the differential Chapman--Kolmogorov 
equation
(\ref{KOL_FORWARD}) 
leads to the master equation for a deterministic Markov process
\begin{equation}
\frac{\partial}{\partial t} T(x,t|x',t') =
 - \sum_i \frac{\partial}{\partial x_i}
     \left(g_i(x)T(x,t|x',t')  \right).
\end{equation}
In statistical physics the above equation is called the Liouville 
equation.
 
The Liouville equation is the starting point for the microscopic 
description of matter for classical as well as for quantum 
mechanical systems. It is one of the fundamental equations of 
statistical physics. 

\subsection{Example: Classical statistical mechanics}
In order to give an example of the occurrence
of the Liouville equation we consider a closed classical system
with $N$ degrees of freedom, e.g., $N$ particles in a 
three--dimensional box. We know from classical mechanics, that the 
state  of such a system is completely specified by the set of $6N$
independent variables $\vec{p}^N=(\vec{p}_1, \ldots, \vec{p}_N)$ 
and $\vec{q}^N=(\vec{q}_1, \ldots, \vec{q}_N)$, where $\vec{p}_i$
and $\vec{q}_i$ denote the momentum and the position of the 
$i$--th particle. 

If the system is Hamiltonian (\cite{ARNOLD}), 
i.e., if we can define a Hamiltonian
$H(\vec{p}^N,\vec{q}^N)$, then the time evolution of the momentum 
and of the position of the particles is given by Hamilton's 
equations of motion
\begin{eqnarray*}
\frac{d}{dt} \vec{p}_i &=& - \frac{\partial H}{\partial \vec{q}_i} 
             \\
\frac{d}{dt} \vec{q}_i &=&  \frac{\partial H}{\partial \vec{p}_i} 
.
\end{eqnarray*}

In a real physical system it is not possible to specify exactly 
the state of the system. There is always some uncertainty in the 
initial conditions. Therefore, we regard $(\vec{p}^N,\vec{q}^N)$
as a stochastic variable which is initially distributed according 
to the joint probability density $P^N(\vec{p}^N,\vec{q}^N,0)$. 
The dynamics of this probability distribution
is described by the following 
Liouville equation
\begin{equation*}
\frac{\partial}{\partial t} P^N =  {\cal{A}}_L^{\dagger} P^N,
\end{equation*}
where
\begin{equation*}
{\cal{A}}_L^{\dagger} = -\sum_{i=1}^N \left( 
      \frac{\partial H}{\partial \vec{p}_i} 
            \cdot \frac{\partial}{\partial \vec{q}_i} 
         - \frac{\partial H}{\partial \vec{q}_i} 
             \cdot \frac{\partial}{\partial \vec{p}_i}.
      \right)
\end{equation*}
The Liouville equation is often written in the following form
\begin{equation*}
i \frac{\partial}{\partial t} P^N(\vec{p}^N,\vec{q}^N,t) =  
  {\cal{L}} P^N(\vec{p}^N,\vec{q}^N,t),
\end{equation*}
where
\begin{equation*}
{\cal{L}} =  i {\cal{A}}_L^{\dagger}.
\end{equation*}
The operator ${\cal{L}}$ is called the Liouville operator.
If the probability disribution at time $t=0$ is known the above 
Liouville equation may be integrated formally  
to find the probability density
at later times $t$
\begin{equation*}
P^N(\vec{p}^N,\vec{q}^N,t) = \exp(-i {\cal{L}}t)  
P^N(\vec{p}^N,\vec{q}^N,0).
\end{equation*}
The Liouville equation is the starting point for the evaluation of 
probability distributions in statistical mechanics. Extensive use 
of the Liouville equation is done in kinetic theory. From the 
Liouville equation it is possible to derive a hierarchy of 
equations for probability densities, the so--called 
BBGKY--hierarchy from which kinetic equations may be derived
(\cite{REICHL}).


\section{The master equation}
Let us now introduce jump processes (\cite{DAVIES,FELLER}).
We consider a system in a given state $x$.  
In order to characterize a jump process, i.e., a process
in which the system undergoes sudden discontinuous changes of its state,
we have to specify the probability for 
the system to remain in $x$ during the time interval $dt$ 
\begin{equation*}
(1-\lambda(x) dt)
\end{equation*}
and the probability that the system jumps from state $x$ to state
$x'$ during the time interval $dt$ 
\begin{equation*}
\lambda(x) Q(x',x) dt,
\end{equation*}
where
\begin{equation}
\label{NORM_Q}
\int dx' Q(x',x) =1.
\end{equation}
Then,
\begin{equation*}
Ef(x(dt+t)) = (1-\lambda(x)dt) f(x)
   + \lambda(x) dt \int dx'f(x') Q(x',x).
\end{equation*}
From the definition of the generator we obtain immediately the
generator of the jump process
\begin{equation}
\label{GEN_JUMP}
{\cal{A}}_M f(x) = \lambda(x) \int dx' \left( f(x') 
-f(x)\right)Q(x',x),
\end{equation}
where we made use of Eq. (\ref{NORM_Q}).
Again, in order to derive the Kolmogorov forward equation we have 
to construct the adjoint operator to ${\cal{A}}_M$.
We start from Eq. (\ref{KOL_FORWARD_0})  
\begin{equation}
\label{KOL_FORWARD_J0}
\int dx \left[ {\cal{A}}_M(t) f(x) \right] T(x,t|x',t') =
  \int dx f(x) \frac{\partial}{\partial t} T(x,t|x',t').
\end{equation}
and insert the generator (\ref{GEN_JUMP}) into the left--hand side 
of the above equation
\begin{eqnarray*}
\lefteqn{\int dx \left[ {\cal{A}}_M(t) f(x) \right] T(x,t|x',t')} 
\\
&=& \int dx \left[ \lambda(x)
       \int dx'' \left(f(x'') -f(x) \right) Q(x'',x) \right] T(x,t|x',t') 
       \\
& = & \int dx \int dx'' \lambda(x) f(x'') Q(x'',x) T(x,t|x',t') \\
& & - \int dx \int dx'' \lambda(x) f(x) Q(x'',x) T(x,t|x',t').
\end{eqnarray*}
By renaming $x \rightarrow x''$ and $x'' \rightarrow x$  in the first line of the
above equation we get
\begin{eqnarray}
\label{ADJOINT_JUMP}
\lefteqn{\int dx  f(x) \int dx'' \lambda(x'')  Q(x,x'') T(x'',t|x',t')}
     \nonumber \\
& & - \int dx f(x) \int dx'' \lambda(x)  Q(x'',x) T(x,t|x',t') \nonumber \\
& \equiv & \int dx f(x) \left[ {\cal{A}}_M^{\dagger}(x) 
          T(x,t|x',t')      \right]
\end{eqnarray}
From Eq. (\ref{KOL_FORWARD_J0}) and from Eq. (\ref{ADJOINT_JUMP}) 
we conclude that the differential Chapman--Kolmogorov equation 
of a jump process
reads
\begin{eqnarray} 
\label{KOL_FORWARD_J1}
\lefteqn{\frac{\partial}{\partial t} T(x,t|x',t') =} \nonumber \\ 
&& \int dx'' \lambda(x'') Q(x,x'') T(x'',t|x',t')
 - \int dx'' \lambda(x) Q(x'',x) T(x,t|x',t').
\end{eqnarray}
Because of Eq. (\ref{NORM_Q}) we can write the above equation also 
in the form
\begin{eqnarray} 
\label{KOL_FORWARD_J2}
\lefteqn{\frac{\partial}{\partial t} T(x,t|x',t') =} \nonumber \\ 
&& \int dx'' \lambda(x'') Q(x,x'') T(x'',t|x',t')
 - \lambda(x) T(x,t|x',t').
\end{eqnarray}
Usually the differential Chapman--Kolmogorov equation for a jump process is 
written in a more suggestive form. To this end we introduce the 
total transition rate pro time unit for a transition from state $x'$ 
into  state $x$ to occur
\begin{equation*}
w(x,x') = \lambda(x') Q(x,x')
\end{equation*}
and write the differential Chapman--Kolmogorov equation 
for a jump process in its final form
\begin{equation}
\label{MASTER_JUMP}
\frac{\partial}{\partial t} T(x,t|x',t') =
 \int dx'' \left( w(x,x'') T(x'',t|x',t')
 - w(x'',x) T(x,t|x',t') \right).
\end{equation}
The above equation is called the master equation. 
The name master equation appears for the first time in a paper by
Nordsieck, Lamb and Uhlenbeck (\cite{NORDSIECK}). 
It was chosen to denote an equation from which
all relevant equations and results can be derived. 

In the physical literature Eq. (\ref{MASTER_JUMP}) is written in 
the simplified form
\begin{equation}
\label{MASTER_JUMP_P}
\frac{\partial}{\partial t} P(x,t) =
 \int dx'' \left( w(x,x'') P(x'',t)
 - w(x'',x) P(x,t) \right).
\end{equation}
This equation has the following meaning (\cite{VAN_KAMPEN}). Take a 
time $t'$ and a state $x'$ and consider 
the solution of Eq. (\ref{MASTER_JUMP_P})
for $t \ge t'$ with the initial condition $P(x,t') = 
\delta(x-x')$. This solution is the conditional transition 
probability $T(x,t|x',t')$ of the Markov process for each choice
of $x'$ and $t'$. It is important to keep in mind that 
Eq. (\ref{MASTER_JUMP_P})
is always to be interpreted as an equation for $T$ and not for 
$P$.

In the above form the physical meaning of the master equation is 
also evident. The master equation is a balance equation for the
probability to find the system in some state. The first term in 
the master equation describes the gain of state due to transitions 
from the other states. The second term is the loss due to the 
transitions from the given state into the others. Evidently,
the term with $x=x'$ does not contribute to the integral.

\subsection{Example: Radioactive decay}
As a very simple example we consider the master equation 
description of radioactive decay. To this end let
$P(n,t)$ be the probability density to find $n$ radioactive nuclei
at time $t$. The probability for a nucleus to decay in unit time will
be denoted by $\gamma$. Since the number of nuclei is integer,
the master equation for the time evolution of $P(n,t)$ will assume 
the following discrete form
\begin{equation}
\label{MASTER_JUMP_P_N}
\frac{\partial}{\partial t} P(n,t) =
 \sum_{n'} \left( w(n,n') P(n')
 - w(n',n) P(n,t) \right).
\end{equation}
It is clear that the transition probability per unit time is given 
by
\begin{equation*}
w(n,n') = \gamma n' \delta(n,n'-1) \;\;\; (n \neq n').
\end{equation*}
Substitution of the above $w(n,n')$ into the master equation
(\ref{MASTER_JUMP_P_N}) yields the master equation of radioactive 
decay
\begin{equation}
\frac{\partial}{\partial t} P(n,t) = 
\gamma (n+1) P(n+1,t) - \gamma n P(n,t).
\end{equation}
The above equation has to be solved for the initial condition
$P(n,0)= \delta(n,n_0)$. 
We will learn in the next chapters how such a master equation may 
be solved numerically. Here, it is interesting to establish the
relation between the master equation and the macroscopic 
description in terms of differential equations we already met in 
the introduction.
To this end we consider
\begin{eqnarray*}
\sum_{n=0}^{\infty} n \dot{P}(n,t) &=&
       \gamma \sum_{n=0}^{\infty} n(n+1)P(n+1) 
         - \gamma  \sum_{n=0}^{\infty} n^2 P(n) \\
  &=& \gamma \sum_{n=0}^{\infty} (n-1)n P(n) 
         - \gamma  \sum_{n=0}^{\infty} n^2 P(n) \\
  & = & - \gamma \sum_{n=0}^{\infty} n P(n) .
\end{eqnarray*}
Thus we have found the following dynamical equation for the 
average of the stochastic 
variable $N(t)$
\begin{equation}
\frac{d}{dt} \langle N(t) \rangle = - \gamma \langle N(t) \rangle.
\end{equation}
Note thate the mean value of the stochastic process obeys the 
differential equation for the concentration. It is clear that the 
above equation has the following solution for the initial value
$\langle N(0)\rangle = n_0$
\begin{equation*}
\langle N(t) \rangle = n_0 \exp(-\gamma t).
\end{equation*}

The master equation for the radioactive decay is a simple example 
of the master equation for a one--step process. We will consider
such stochastic processes in more detail in later chapters.

\section{The Fokker--Planck equation}
In this section we want to derive the Fokker--Planck equation
(\cite{RISKEN}),
which is a special type of master equation (\cite{VAN_KAMPEN})
in the limit of small jumps.
We begin by expressing the transition probability $w$ as a 
function of the size $r$ of the jump and of the starting point
\begin{equation*}
w(y,y') = w(y';r); \;\;\; r=y-y'.
\end{equation*}
The master equation (\ref{MASTER_JUMP_P}) then reads
\begin{equation}
\label{MASTER_SMALL_JUMP}
\frac{\partial}{\partial t} P(y,t) =
  \int dr w(y-r;r)P(y-r,t) - \int dr w(y;-r)P(y,t).
\end{equation}
In order to consider the limit of small jumps essentially
two assumptions will be needed. The first is that
the function
$w(y;r)$ will be a sharply peaked function of $r$ and will
vary slowly with $y$. To be more precise we assume that a $\delta >0$
exists such that
\begin{eqnarray*}
w(y';r)  & \approx & 0 \;\;\; \text{for} \;\;\; |r|> \delta \\
w(y'+\Delta y;r) & \approx & w(y';r) \;\;\; \text{for} |\Delta y| < 
\delta.
\end{eqnarray*}
The second assumption is that the solution $P(y,t)$ of the master equation 
in this limit will be a slowly varying function of $y$.

If these assumptions hold it is safe to expand the shift from $y$
to $y-r$ in the first integral in 
Eq. (\ref{MASTER_SMALL_JUMP}) in a Taylor series
\begin{equation}
\label{MASTER_EXPAND}
\frac{\partial}{\partial t} P(y,t) = \sum_{\nu=0}^{\infty}
   \frac{(-1)^{\nu}}{\nu!} \left( \frac{\partial}{\partial y} 
   \right)^{\nu}
    \left\{  a_{\nu}(y) P(y,t) \right\}
    - P(y,t) \int_{-\infty}^{\infty} dr w(y;-r),
\end{equation}
where we have defined
\begin{equation*}
a_{\nu}(y) = \int_{-\infty}^{\infty} dr r^{\nu} w(y;r).
\end{equation*}
Since the zeroth term in the sum and the second term of 
Eq. (\ref{MASTER_EXPAND}) cancel the small jumps expansion 
of the master equation reads
\begin{equation}
\frac{\partial}{\partial t} P(y,t) = \sum_{\nu=1}^{\infty}
   \frac{(-1)^{\nu}}{\nu!} \left( \frac{\partial}{\partial y} 
   \right)^{\nu}
    \left\{  a_{\nu}(y) P(y,t) \right\}.
\end{equation}
The above expansion is called the Kramers--Moyal expansion. 
Formally, we can write
\begin{equation}
\frac{\partial}{\partial t} P(y,t) = {\cal{A}}_{KM}^{\dagger}(y)
         P(y,t),
\end{equation}
where we introduced the adjoint generator
\begin{equation*}
{\cal{A}}_{KM}^{\dagger}(y) = \sum_{\nu=1}^{\infty}
   \frac{(-1)^{\nu}}{\nu!} \left( \frac{\partial}{\partial y} 
   \right)^{\nu}
     a_{\nu}(y).
\end{equation*}
It is easy to show by partial integration that the corresponding
generator  reads
\begin{equation*}
{\cal{A}}_{KM}(y) = \sum_{\nu=1}^{\infty}
   \frac{1}{\nu!}  a_{\nu}(y)
    \left( \frac{\partial}{\partial y} 
   \right)^{\nu}.
\end{equation*}
It is clear that dealing with the {\em Kramers--Moyal} expansion will 
not be easier than dealing with the original master equation.

A particularly interesting and useful approximation to a jump process is 
obtained by keeping only terms up to the second order in $\nu$
\begin{equation}
\label{FOKKER_PLANCK}
\frac{\partial}{\partial t} P(y,t) =
-\frac{\partial}{\partial y} \left\{ a_1(y) P(y,t) \right\}
   + \frac{1}{2} \frac{\partial^2}{\partial y^2} 
      \left\{ a_2(y) P(y,t) \right\}.
\end{equation}
The above equation is called the {\em Fokker--Planck equation}.
The first term on the right--hand side is usally called the drift
term since it is essentially Liouvillian. The second term
on the right--hand side is the diffusion term. In a later chapter
we will learn kow to deal with this equation.

Let us mention at the end of this section the important special case 
of vanishing drift, i.e. $a_1=0$ and $a_2=1$. The generator 
\begin{equation}
\label{WIENER_GENERATOR}
{\cal{A}}_W = {\cal{A}}_W^{\dagger} = \frac{1}{2} 
     \frac{\partial^2}{\partial y^2}
\end{equation}
generates a so--called Wiener process.

\subsection{Brownian motion}
As an example of the application of the Fokker--Planck
equation we consider Brownian motion. A heavy particle is immersed 
in a fluid of light particles. The large particle and the small 
molecules collide in a random fashion. These 
collisions induce sudden variations of the velocity of the 
Brownian particle. As a consequence the position $X$ of the Brownian 
motion may be described by a Markov 
process, if observed on a time scale which is larger then the 
autocorrelation time of the velocity, i.e. many collisions have 
occurred between two observations. If we regard, for simplicity a 
one--dimensional 
situation, the Brownian particle makes random jumps back and forth 
along the $x$--axis. The jumps may have any length, but the
probability for large jumps falls off rapidly. 
It is reasonable to assume that these probabilities are 
symmetrical and independent of the starting point.

Hence we have
\begin{equation*}
a_1 = \frac{\langle \Delta X \rangle}{\Delta t} = 0
\end{equation*}
and
\begin{equation}
a_2 = \frac{\langle (\Delta X)^2\rangle}{\Delta t}= \text{const}.
\end{equation}
The Fokker--Planck equation of Brownian motion thus reads
\begin{equation}
\label{BROWNIAN}
\frac{\partial}{\partial t} P(X,t) = \frac{a_2}{2} 
    \frac{\partial^2}{\partial x^2} P(x,t).
\end{equation}
The above equation is a diffusion equation and we immediately
identify the diffusion constant to be
\begin{equation}
D = \frac{\langle (\Delta X)^2\rangle}{2 \Delta t}.
\end{equation}
The above relation is known as the Einstein relation. It relates 
the diffusion constant with the microscopic jumps of the Brownian 
particle.

Let us now consider an ensemble of Brownian particles, which are
concentrated in $X=0$ at time $t=0$. Their position at time $t$ is 
a stochastic process $X(t)$, whose transition probability is given 
by (\ref{BROWNIAN}). This process, as we know, is a Wiener 
process. The probability density of the Brownian particles  with 
initial condition $P(x,0) = \delta(x)$ at time $t>0$ is easily 
found to be
\begin{equation}
P(x,t) = \frac{1}{\sqrt{4 \pi D t}} \exp(-x^2/4Dt).
\end{equation}
This is a gaussian with maximum at $x=0$ and whose standard variation grows
as
\begin{equation}
\langle X^2(t) \rangle = 2 D t.
\end{equation}
Hence, the width of the distribution increases with the square 
root of the time. A behaviour which we already know from our 
discussion of random walks.

\section{Beyond this chapter}
In this chapter we have introduced the key concepts which arise
in the theory of stochastic processes. The mathematical more 
interested reader will find a mathematical more precise introduction 
in the classical books by Feller
(\cite{FELLER}) and by Doob (\cite{DOOB}). 

Students interested in the physical applications of stochastic processes
should read the classical paper by Chandrasekhar 
(\cite{CHANDRASEKHAR}) which is reprinted in \cite{WAX}.

In this chapter we have only introduced the formal description,
i.e., the kinematics of Markov processes. We have not discussed
how Markov processes can be derived from some microscopic picture.
A survey of such techniques can be found in the book by Oppenheim 
et al. (\cite{OPPENHEIM}).


\bibliographystyle{peter}
\bibliography{V_98}

%%%%%%%%%%%%%%%%%%%%%%%%%%%%%%%%%%%%%%%%%%%%%%%%%%%%%%%%%%%%%%%%%%
%%%%%%%% Ende von Kap. 4 %%%%%%%%%%%%%%%%%%%%
%%%%%%%%%%%%%%%%%%%%%%%%%%%%%%%%%%%%%%%%%%%%%%%%%%%%%%%%%%%%%%%%%%%%%


\chapter{Monte Carlo Methoden in der statistischen Mechanik}
Metropolis; Ising; Finite-Size Effects; Random Walks; SAW (?)
Simulated annealing; travelling salesman;

\chapter{Non-equilibrium MC}
Chemische Reaktionen, Diffusion; Reaktions-Diffusion, Turbulenz.

\chapter{Brownsche Dynamik Simulationen}
Omega-entwicklung; SDE;


\chapter{Rest}
stochastische Resonanzen; Muster Erkennung; random Walks;

\chapter{Stochastische Wellenfunktionsmethoden}
