%%%%%
%%%%% Chapter 3
%%%%%
\chapter{Simple Sampling of Probability distributions Using Random numbers}


%%%%%%%%%%%%%%%%%%%%%%%%%%%%%%%%%%%%%%%%%%%%%%%%%%%%%
\section{Die Transformationsmethode}
Lit: D.T. Gillepsie, a theorem for physicists in the theory of 
random variables, Am. J. Phys. 51 (1983) 520-533.

%%%%%%%%%%%%%%%%%%%%%%%%%%%%%%%%%%%%%%%%%%%%%%%%%%%%%
\section{Importance Sampling}

%%%%%%%%%%%%%%%%%%%%%%%%%%%%%%%%%%%%%%%%%%%%%%%%%%%%%
\section{Monte Carlo Integration}


Programme:
\begin{itemize}
\item Monte--Carlo Bestimmung von $\pi$
\item Monte-Carlo Bestimmung von $e$
\item Monte-Carlo Integration
\end{itemize}

Literatur:
\begin{itemize}
\item Karlen Skript
\item Stauffer
\item Knuth
\end{itemize}

Programme:
\begin{itemize}
\item Test von Zufallszahlen Generatoren (1d, 2D, 3D)
\end{itemize}


%%%%% trandom1 %%%%%%%%%%%%%%%%
\begin{verbatim}
% trandom1 - Program to demonstrate the generation of random numbers
% using the linear congruential method
clear; help trandom1 %clear the memory and print header
seed = input('Enter the seed (1) - ');
m = input('Enter the modulus (8) - ');
a = input('Enter the multiplier (5) - ');
c = input('Enter the increment (1<=c<7) - ');
% Set starting value
R(1) = seed;
% Generate vector of m random numbers
for j=1:m
   R(j+1) =floor(rem(a*R(j)+c,m));
end
%R=R/M;
disp('The generated series is:')
disp(R)
plot(R,'x')
title('The series of generated random numbers');
xlabel('Term, i'); ylabel('Value');
\end{verbatim}
%%%%%%%%%%%%%%%%%%%%%%%%%%%%%%%%%%%%%%%%%%%%%%%%%%%%%%%%%%%%%%%

\begin{verbatim}
% trandom2 - Program to test random number generators
% The program makes use of the random number generator random1
clear; help trandom2; % clear memory and print header
% Enter dimension of random vector
n= input('Enter value of n-'); % 
% generate random vector
R1=random1(n);
% plot random vector
plot(R1)
title('random numbers'); xlabel('random variable');
disp('Histogram: press any keyboard key');
pause;
% plot histogram of random numbers
x=(0.05:0.1:0.95);
[m,xout] = hist(R1,x);
bar(xout,m)
xlabel('random number');
title('1D distribution: Histogram');
disp('2D plot: press any keyboard key');
pause;
% 2D plot: correlation of consecutive numbers
R2x=R1(1:2:n-1);
R2y=R1(2:2:n);
plot(R2x,R2y,'x')
title('2d distribution')
disp('3D plot: press any keyboard key');
pause;
% 3D plot: correlation of consecutive numbers
R3x=R1(1:3:n);
R3y=R1(2:3:n);
R3z=R1(3:3:n);
%R3=random1(n,3);
plot3(R3x,R3y,R3z,'x')
title('3D distribution');xlabel('Random number');
\end{verbatim}

%%%%%%%%%%%%%%%%%%%%%%%%%%%%%%%%%%%%%%%%%%%%
\begin{verbatim}
function R = random1(n)
% function to generate random numbers
a=65539;
M=2^(31)-1;
R(1) = 12345678;
for j=1:n-1
%   for i=1:n-1
   R(j+1) =floor(rem(a*R(j),M));
%end
end
R=R/M;
\end{verbatim}







%%%%%%%%%%%%%%%%%%%%%%%%%%%%%%%%%%%%%%%%%%%%%%%%%%%%%
%%%%%%%%%%%%%%%%%%%%%%%%%%%%%%%%%%%%%%%%%%%%%%%%%%%%%
%%%%%%%%%%%%%%%%%%%%%%%%%%%%%%%%%%%%%%%%%%%%%%%%%%%%%
%%%%%%%%%%%%%%%%%%%%%%%%%%%%%%%%%%%%%%%%%%%%%%%%%%%%%
%%%%%%%%%%%%%%%%%%%%%%%%%%%%%%%%%%%%%%%%%%%%%%%%%%%%%
%%%%%%%%%%%%%%%%%%%%%%%%%%%%%%%%%%%%%%%%%%%%%%%%%%%%%
%%%%%%%%%%%%%%%%%%%%%%%%%%%%%%%%%%%%%%%%%%%%%%%%%%%%%
\chapter{Stochastische Prozesse}
Master-Gleichungen

\chapter{Monte Carlo Methoden in der statistischen Mechanik}
Metropolis; Ising; Finite-Size Effects; Random Walks; SAW (?)
Simulated annealing; travelling salesman;

\chapter{Non-equilibrium MC}
Chemische Reaktionen, Diffusion; Reaktions-Diffusion, Turbulenz.

\chapter{Brownsche Dynamik Simulationen}
Omega-entwicklung; SDE;


\chapter{Rest}
stochastische Resonanzen; Muster Erkennung; random Walks;

\chapter{Stochastische Wellenfunktionsmethoden}

