\documentclass[a4paper,12pt]{article}

\usepackage[german]{babel}
\usepackage[latin1]{inputenc}
\usepackage[T1]{fontenc}

\parindent0cm

\begin{document}
\begin{center}
\underline{\LARGE Contents of the JavaSimulation CD}
\end{center}

Comment: There are some files in the tar.gz or .Z format used under
UNIX Systems. We have unpacked them all on the CD ROM, so you can 
access them also from Windows. Some .zip files have not been unpacked
because both operating systems can unpack and process zip files.

What you really need:
\begin{itemize}
\item WinZip: Most of the files on the CD ROM need an Unzip Utility.
  We have copied the WinZip Utility on the CD ROM in the Windows95 directory.
  Install this first if you do not have any ZIP Utility available.
  For Windows use CD-ROM/Windows95/wz32v63.exe.
\item Java Development Kit (JDK): jdk 1.1.7 for Windows 95/NT4 or for Linux,
  there are the Debian files (Debian 2.0/Hamm) and the SUSE V5.3 Files
  in the CD-ROM/Windows95/SUN/jdk117-win32.exe and the
  CD-ROM/Linux/SuSe/d1 or CD-ROM/Linux/Debian directory. Follow Linux
  installation instructions and on Win just execute the file or unpack them.
\item Netscape Communicator V4.07: again there is a version for Windows95/NT 
  and a Linux Version (for both Debian and SUSE). Please read the SUSE
  documentation on how to install the Netscape Navigator. The files are:
  CD-ROM/Windows95/cp32e407.exe, 
  CD-ROM/Linux/communicator-v407-export.x86-unknown-linux2.0\_libc5.tar.gz
  for the Communicator and CD-ROM/Linux/Debian/netscape4\_4.0-12.deb
  for the Debian Distrib. Use the Netscape.deb file to install running Debian,
  for the SUSE distribution, please read the documentation.
\item Emacs/Xemacs: If you want to use Emacs or Xemacs for writing and
  editing programs, you can use the supplied versions of the programs.
  Either the ge (Emacs) or the xe (xemacs) files in CD-ROM/Linux/SuSe/e1,
  the analogous files for the debian in the debian dir. For Windows use
  CD-ROM/Windows95/emacs-20.3.1-bin-i386 directory (emacs).
  Xeamcs is currently under development for Windows and not available for
  easy use. Please use emacs.
\item Ghostview/Ghostscript: For viewing PS files of the documentation, you
  need a postscript viewer. You can install 
  CD-ROM/Windows95/gsv26550.exe for Windows. Linux probably already has
  an installed version.
\item Acrobat Reader: For viewing PDF Files, use CD-ROM/Windows95/ar32d301.exe
  or CD-ROM/Linux/acroread\_linux\_301.tar.gz for Linux.
\item Java Workshop / Java Studio: Install CD-ROM/Windows95/SUN/SETUPJWS.EXE 
  (Workshop) or CD-ROM/Windows95/SUN/SETUPJS.EXE (Studio). Under Linux
  you can use the Workshop, we do not know how to run the Studio.
  Use the files CD-ROM/Linux/jws2.0.intel-S2.tar.Z and
  CD-ROM/Linux/jws\_linux.tar.gz for the Workshop and
  CD-ROM/Linux/JS1.0.intel-S2.tar.Z for the Studio.
\item SciVis the visualization java program: use
  CD-ROM/Java/programs/scivis11.tar.gz for all platforms and the examples
  in CD-ROM/Java/programs/scivis.samples.tar.gz.
\item Documentation/Tutorials/FAQs: See the large directory 
  CD-ROM/Java/Documentation.
\item Examples: See the CD-ROM/Java/Examples directory.
\end{itemize}

\end{document}
