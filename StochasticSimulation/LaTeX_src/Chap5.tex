\chapter{Stochastic differential equations}
In the previous section we have derived an exact simulation algorithm
for the generation of trajectories of the Ornstein--Uhlenbeck process.
The ``exact'' update formula was
\begin{equation*}
X(t+\Delta t) = X(t) \exp(-q \Delta t) +
  \left[ \frac{D}{2q}(1-\exp(-2q\Delta t)) \right]^{1/2} \xi(t),
\end{equation*}
where we have now written $\xi(t)$ to stress the fact that at each
time step $t$ we have to draw another Gaussian distributed random 
number. The update formula is exact in the sense that it holds for
arbitrary values of $\Delta t$.

However, it will turn out to be convenient to have an update formula
which works for small values of $\Delta t$. To this end we expand the
exact update formula to first order in $\Delta t$ and obtain
\begin{eqnarray}
X(t+\Delta t)& =& X(t) (1- q \Delta t) + 
   \left[ \frac{D}{2q} (2q \Delta t) \right]^{1/2} \eta(t) \nonumber
   \\
\label{SDE_APPROX} 
  & = & X(t) - qX(t) \Delta t + \sqrt{D} \sqrt{\Delta t} \eta(t).
\end{eqnarray}
In the limit $\Delta t \rightarrow 0$ this approximate update formula
turns exact.  We recognize immediately that the stochastic increment in
this discretized version of the Ornstein--Uhlenbeck process scales
with the square root of the time increment $\Delta t$.

Note that in deriving the above discretized update formula we have 
intentionally omitted the terms linear in $\Delta t$ stemming from 
the expansion of the factor in front of the stochastic term. In 
doing so we have achieved that the update formula has an important
property. Namely, it is selfconsistent in the following sense.
Let us apply the above formula twice, starting from,
\begin{equation*}
X(t+2\Delta t) = X(t+\Delta t) 
 - qX(t+ \Delta t) \Delta t + \sqrt{D} \sqrt{\Delta t} \xi(t+\Delta t)
\end{equation*}.
Inserting (\ref{SDE_APPROX}) we immediately obtain keeping 
terms up to first order in $\Delta t$
\begin{equation*}
X(t+2 \Delta t) = X(t) - q X(t) 2 \Delta t +
   \sqrt{D} \sqrt{\Delta t} [\xi(t) + \xi(t+\Delta t)].
\end{equation*}
Since $\xi(t)$ and $\xi(t+\Delta)$ are statistically independent
Gaussian stochastic processes we have
\begin{equation*}
\xi(t) + \xi(t+\Delta t) = {\bf N}(0,1) + {\bf N}(0,1) =
  {\bf N}(0,2) = \sqrt{2} {\bf N}(0,1),
\end{equation*}
so that we finally have
\begin{equation*}
X(t+2 \Delta t) = X(t) - q X(t) 2 \Delta t +
   \sqrt{D} \sqrt{2 \Delta t} \xi(t).
\end{equation*}
This selfconsistency of the discretized stochastic differential 
equation expresses essentially the fundamental properties
of the propagator of a Markov process as they are defined in
the Chapman--Kolmogorov equation.


Due to the presence of a stochastic term, the Gaussain stochastic
process $\xi(t)$ the above equation is a typical example of a 
so--called {\em stochastic differential equation} (SDE). It is the aim
of this section to introduce into some  peculiarities of stochastic
differential equations. In particular we will also show the
equivalence of stochastic process defined in terms of stochastic differential
equations and in terms of Fokker--Planck equations.

The above expression is a special case of the  {\em standard form} of
a stochastic differential equation (some times stochastic
differential equations are also called {\em Langevin} equations):
\begin{equation}
\label{SDE_LANGEVIN_DISCR}
X(t+dt) = X(t) + A(X(t),t)dt + D(X(t),t)^{1/2} \xi(t) dt^{1/2},
\end{equation}
where we have replaced $\Delta t$  by $dt$ to stress the infinitesimal
character of the above equation. The term proportional to $dt$ is
called the {\em drift term}, whereas the term proportional to
$\sqrt{dt}$ is called the diffusion term. 

The above definition of the stochastic process $X(t)$
in terms of a stochastic differential equation  cleary shows that
the stochastic process $X(t)$ is continuous, but, in general, not 
differentiable. This can be seen by writing
Eq. (\ref{SDE_LANGEVIN_DISCR}) as
\begin{equation*}
\frac{X(t+dt) - X(t)}{dt} = A(X(t),t) + \frac{D^{1/2}(X(t),t) \xi(t)}
                                              {\sqrt{dt}}.
\end{equation*}
Obviuosly, the limit $dt \rightarrow 0$ of the above equation does not
exist, unless $D\equiv 0$. Thus, a purely stochastic Markov process
is everywhere continuous but nowhere differentiable. Nevertheless it is
custumary to ``pretend'' \cite{GILLESPIE} that $dx/dt$ exists even for
nonvanishing $D$. In fact we know that we can write
\begin{equation*}
\frac{\xi(t)}{\sqrt{dt}} = \frac{1}{\sqrt{dt}} {\bf N}(0,1) =
{\bf N}(0,1/dt).
\end{equation*}
So, we may define a Gaussian {\em white noise process} as
\begin{equation*}
\eta(t) \equiv \lim_{dt \rightarrow 0}  {\bf N}(0,1/dt).
\end{equation*}
With the help of the above definition, we can now formally write
\begin{equation*}
\frac{d}{dt} X(t) = A(X(t),t) + \sqrt{D}(X(t),t) \eta(t).
\end{equation*}
This equation is called the white noise form of the Langevin equation.
The white noise process introduced above does have the following
averaged properties:
\begin{eqnarray*}
\langle \eta(t) \rangle &=& 0 \\
\langle \eta(t) \eta(t') \rangle &=& \delta(t-t'),
\end{eqnarray*}
which satisfy the requirement of no correlation at different
times. Note, that the white noise process  $\eta$ has
infinite variance.

\section{The Langevin equation and Brownian motion}
In 1908 Langevin considered the problem of the dynamical 
description of Brownian motion. He suggested that the equation of 
motion of a Brownian particle with mass $m=1$ be described by the
following differential equation for the velocity $V$
\begin{equation}
\label{LANGEVIN}
\frac{d}{dt} V = -\gamma V + L(t),
\end{equation}
where the terms on the right hand--side of the above equation 
model the forces which the surrounding molecules excerpt on the 
Brownian particle. Since these forces are unknown in detail the 
following assumptions were postulated. The Brownian particle 
moving in the fluid of surrounding particles feels
a dissipative drag force which is proportional to its velocity, $\gamma$  being 
the friction coefficient. Furthermore, the Brownian particle hits
the surrounding particles. These collisions cause irregular 
changes in the velocity of the Brownian particle. Thus, the external force
$L(t)$ is modeled as a zero mean, temporally uncorrelated
randomly fluctuating force. The first two moments of the
stochastic process $L(t)$ are assumed to 
have the following properties
\begin{eqnarray*}
\langle L(t) \rangle &=& 0 \\
\langle L(t) L(t') \rangle &=& \Gamma \delta(t-t').
\end{eqnarray*}

The Langevin equation is the prototype of a stochastic 
differential equation, i.e. of a differential equation whose
coefficients are random functions of the time with some given
statistical properties. 
It is clear that choosing $L(t)=\sqrt{\Gamma} \eta(t)$,
where $\eta(t)$ is a Gaussian white noise process, the langevin 
equation of Brownian motion describes an Ornstein--Uhlenbeck
process.
The stochastic process $V(t)$ is 
completely defined once an initial condition $V(0)=V_0$ is specified.
Its formal solution reads
\begin{equation*}
V(t) = V_0 \exp(-\gamma t) + \exp(-\gamma t) 
    \int_0^t dt \exp(\gamma t') L(t').
\end{equation*}
Taking the average over an ensemble of Brownian particles all
having the same initial condition we find for the mean value of 
the velocity
\begin{equation*}
\langle V(t) \rangle = V_0 \exp(-\gamma t),
\end{equation*}
where we made use of the statistical properties of the Langevin 
force $L(t)$. Accordingly, the second moment of the velocity field
is found to be
\begin{eqnarray*}
\langle V^2(t) \rangle &=& V_0^2 \exp(-2\gamma t)
          + \exp(-2 \gamma t) \int_0^t dt'' \int_0^t dt'
             \exp(\gamma (t' + t'')) \langle L(t') L(t'') \rangle 
             \\
          &=& V_0^2 \exp(-2\gamma t) + \frac{\Gamma}{2 \gamma}
             [1-\exp(-2 \gamma t)].
\end{eqnarray*}
Up to now the constant $\Gamma$ was left unspecified. From 
equilibrium statistical physics (theorem of equipartition of energy) 
we  expect that for long times 
\begin{equation*}
\langle V^2(t\rightarrow \infty) \rangle = kT.
\end{equation*}
Hence we have
\begin{equation}
\label{FDT}
\Gamma = 2 \gamma kT
\end{equation}
and we have established a relation between the attrition 
coefficient $\gamma$ and the random fluctuations.
Eq. (\ref{FDT}) is a simple version of the so--called
fluctuation--dissipation theorem.

\section{The Ito formula}
In the previous sections we have considered Langevin equations 
from a rather intuitive point of view. Here we want to consider
such equations in a more formal setting and in greater generality.
Let us begin by considering the one--dimensional stochastic 
differential equation for the stochastic variable $X(t)$
\begin{equation*}
\dot{X}(t) = a(X(t),t) + \eta(t),
\end{equation*}
where $\eta(t)$ is a normal white noise process.
It follows from the above equation that up 
to terms of the order $dt$ the increment $dX(t)$
of the stochastic variable $X(t)$ during a time step $dt$
is given by
\begin{equation*}
dX(t) = a(X(t),t)dt + dW(t),
\end{equation*}
where
\begin{equation*}
dW(t) = \int_t^{t+dt} \eta(t') dt'
\end{equation*}


\section{The equivalence of stochastic differential equations 
and of the Fokker--Planck equation}

\section{The numerical integration of stochastic differential 
equations}
\subsection{The Ornstein--Uhlenbeck process}






\subsection{Brownian motion}
As an example of the application of the Fokker--Planck
equation we consider Brownian motion. A heavy particle is immersed 
in a fluid of light particles. The large particle and the small 
molecules collide in a random fashion. These 
collisions induce sudden variations of the velocity of the 
Brownian particle. As a consequence the position $X$ of the Brownian 
motion may be described by a Markov 
process, if observed on a time scale which is larger then the 
autocorrelation time of the velocity, i.e. many collisions have 
occurred between two observations. If we regard, for simplicity a 
one--dimensional 
situation, the Brownian particle makes random jumps back and forth 
along the $x$--axis. The jumps may have any length, but the
probability for large jumps falls off rapidly. 
It is reasonable to assume that these probabilities are 
symmetrical and independent of the starting point.

Hence we have
\begin{equation*}
a_1 = \frac{\langle \Delta X \rangle}{\Delta t} = 0
\end{equation*}
and
\begin{equation}
a_2 = \frac{\langle (\Delta X)^2\rangle}{\Delta t}= \text{const}.
\end{equation}
The Fokker--Planck equation of Brownian motion thus reads
\begin{equation}
\label{BROWNIAN}
\frac{\partial}{\partial t} P(X,t) = \frac{a_2}{2} 
    \frac{\partial^2}{\partial x^2} P(x,t).
\end{equation}
The above equation is a diffusion equation and we immediately
identify the diffusion constant to be
\begin{equation}
D = \frac{\langle (\Delta X)^2\rangle}{2 \Delta t}.
\end{equation}
The above relation is known as the Einstein relation. It relates 
the diffusion constant with the microscopic jumps of the Brownian 
particle.

Let us now consider an ensemble of Brownian particles, which are
concentrated in $X=0$ at time $t=0$. Their position at time $t$ is 
a stochastic process $X(t)$, whose transition probability is given 
by (\ref{BROWNIAN}). This process, as we know, is a Wiener 
process. The probability density of the Brownian particles  with 
initial condition $P(x,0) = \delta(x)$ at time $t>0$ is easily 
found to be
\begin{equation}
P(x,t) = \frac{1}{\sqrt{4 \pi D t}} \exp(-x^2/4Dt).
\end{equation}
This is a gaussian with maximum at $x=0$ and whose standard variation grows
as
\begin{equation}
\langle X^2(t) \rangle = 2 D t.
\end{equation}
Hence, the width of the distribution increases with the square 
root of the time. A behaviour which we already know from our 
discussion of random walks.



\bibliographystyle{peter}
\bibliography{V_98}


\chapter{Monte Carlo Methoden in der statistischen Mechanik}
Metropolis; Ising; Finite-Size Effects; Random Walks; SAW (?)
Simulated annealing; travelling salesman;

\chapter{Non-equilibrium MC}
Chemische Reaktionen, Diffusion; Reaktions-Diffusion, Turbulenz.

\chapter{Brownsche Dynamik Simulationen}
Omega-entwicklung; SDE;


\chapter{Rest}
stochastische Resonanzen; Muster Erkennung; random Walks;

\chapter{Stochastische Wellenfunktionsmethoden}
