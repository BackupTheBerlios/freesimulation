\documentclass[a4paper]{book}
\usepackage[english]{babel}
\usepackage[T1]{fontenc}

\begin{document}

\chapter*{Intro}

\section*{Overview}

The primary aim of the book is to introduce undergraduate and graduate 
students to the theory and practice of stochastic simulation methods.
The book is intended to be a  hands on introduction. 
The theory and the practice are intertwined:  Computational methods are 
best learned by working out examples, therefore we explain in detail how to 
implement the algorithms in the modern programming language Java, which has 
a great 
potential for the future. At the end of each chapter we have given
some exercises to deepen and recapitulate the methods learned and to
address some related problems (Solutions are given in the appendix.).
All algorithms are explained extensively and 
Java programs are always given to study the learned topics.
Java GUIs allow the reader an immediate interactive exploration of
most topics.

Our approach is twofold. On the one side the  stochastic simulation
methods developed in the book are embedded in the context of the
mathematical theory of stochastic processes. The clean but not too formal 
exposition of the mathematical methods allows
the students to generalize the tools and to apply them during their future 
career. We have included a chapter on financial engineering to
demonstrate the flexibility and the importance of the methods outside of
physics. 

On the other side we are well-aware that physics students seeking a
job often are expected to know Java.  Therefore, we introduce the
student step by step into programming with Java. The necessary Java
techniques are explained in the context where they are required.
We start from the canonical ``Hello World'' program and end with an 
introduction to high performance parallel computing not only in Java.
Furthermore, the platform independence of Java and its free
availability are  a great advantage. Students can practice in the class 
and at home on different platforms without having to buy expensive
software. 

The book will be supplemented by a CD ROM, which includes an
electronic version of the book, all Java programs and GUIs, which
are accessible through a web browser.

%% Students learn a language very important for 
%% their future career in industry and education or science.



\section*{Market Need}
Mostly undergraduate students, who are interested in the basics and
applications of Monte Carlo methods are mainly addressed.
The graduate student or the scientist, who is already familiar with 
the basic concepts in simulations in the first few chapters, 
can gain a lot by studying the second part of the book.
These are about modern simulation techniques in non-equilibrium systems, 
many particle systems and parallel computing.
The last two chapters about simulation of open quantum systems and about 
financial problems are devoted to people who want to see how the methods 
described in the first chapters can be applied to modern problems.

The whole book can also be viewed as an extension or companion to the famous ``Numerical Recipes'' book, which concentrates on traditional numerical methods.
But not only the numerical side also the computational side is new in the 
sense, that there is no Numerical Recipes book using Java available -- 
although the book by Davies tries to fill part of this gap.

So to summarize, there seems to be a lack of books about Monte Carlo
methods including non-equilibrium, quantum mechanical and non
physics applications on an undergraduate level, which is formal enough,
but still very applied. Together with an introduction to Java it is, what
we think, a standard book for all people interested in simulations
in statistical physics. We think it will be very appreciated by many
scientists eagerly waiting for a ``scientific'' Java introduction.


\section*{Send In}
Chap 0 to Chap4 !!!

\end{document}

