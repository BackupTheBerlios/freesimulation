\documentclass[a4paper]{book}
\usepackage[english]{babel}
\usepackage[T1]{fontenc}

\begin{document}

\chapter*{Intro}

\section*{Overview}

The primary aim of the book is to introduce undergraduate and graduate 
students to the theory and practice of stochastic simulation methods.
The book is intended to be a  hands on introduction. 
The theory and the practice are interwined:  Computational methods are 
best learned by working out examples, therefore we explain in detail how to 
implement the algorithms in a modern programming language with a great 
potential for the future. At the end of each chapter we have given
some exercises to deepen and recapitulate the methods learned and to
address some connected problems (Extensive 
solutions are given in the appendix.)
All algorithms are explained extensively and 
Java programs are always given to deepen the learned topics.

Our approach is twofold. On the one side the  stochastic simulation
methods developed in the book are embedded in the context of the
mathematical theory of stochastic processes. The clean but not too formal 
exposition of the mathematical methods allows
the students to generalize the tools and to apply them during their future 
career. We have included a chapter on financial engineering to
demonstrate the flexibility and the importance of the methods outside of
physics. 

On the other side we are well-aware that physics students seeking a
job often are expected to know Java.  Therefore, we introduce the
student step by step into programming with Java. The necessary Java
techniques are explained in the context where they are required.
We start from the canonical ``Hello World'' program and end with an 
introduction to high performance parallel computing not only in Java.
Furthermore, the platform independence of Java and its free
availability are  a great advantage. Students can practise in the class 
and at home (using the supplied CD) on 
different platforms without having to buy expensive
software. Students learn a language very important for 
their future career in industry and education or science.



\section*{Market Need}

But also the graduate student or the scientist, who is alreay familiar with the basic concepts in simulations given in the first few chapters, can gain a lot by studying the second part of the book.
These are about modern simulation techniques in non-equilibrium systems, many particle systems and parallel computing.
The last two chapters about simulation of open quantum systems and about financial problems are devoted to people who want to see how the methods described in the first chapters can be applied to these problems.

The whole book can also be viewed as an extension or companion to the famous ``Numerical Recipes'' book, which concentrates on traditional numerical methods.
But not only the numerical side also the computational side is new in the sense, that there is no Numerical Recipes book using Java available, although the book by Davies tries to fill this gap.


\end{document}
