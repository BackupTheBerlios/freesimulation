%%%
%%%    Copyright (c)  2002  Peter Biechele, Francesco Petruccione
%%%
%%%    Permission is granted to copy, distribute and/or modify this document
%%%    under the terms of the GNU Free Documentation License, Version 1.1
%%%    or any later version published by the Free Software Foundation;
%%%    with no Invariant Sections, with the Front-Cover Texts being
%%%    FrontCover.tex (chapter Front Cover), and with the Back-Cover 
%%%    Texts being BackCover.tex (Chapter Back Cover).
%%%    A copy of the license is included in the section entitled "GNU
%%%    Free Documentation License".
%%%
%%%

%%%
%%% This whole file is the Back Cover text !!!
%%%

\chapter*{Back Cover}

%%% No pagenumbers - althopugh we have choosen \pagestyle{empty}
\thispagestyle{empty}

The primary aim of the book is to introduce undergraduate and graduate 
students to the theory and practice of stochastic simulation methods.
The book is intended to be a  hands on introduction. 
The theory and the practice are intertwined:  Computational methods are 
best learned by working out examples, therefore we explain in detail how to 
implement the algorithms in the modern programming language Java, which has 
a great 
potential for the future. At the end of each chapter we have given
some exercises to deepen and recapitulate the methods learned and to
address some related problems (Solutions are given in the appendix.).
All algorithms are explained extensively and 
Java programs are always given to study the learned topics.
Java GUIs allow the reader an immediate interactive exploration of
most topics.

Our approach is twofold. On the one side the  stochastic simulation
methods developed in the book are embedded in the context of the
mathematical theory of stochastic processes. The clean but not too formal 
exposition of the mathematical methods allows
the students to generalize the tools and to apply them during their future 
career. We have included a chapter on financial engineering to
demonstrate the flexibility and the importance of the methods outside of
physics. 

On the other side we are well-aware that physics students seeking a
job often are expected to know Java.  Therefore, we introduce the
student step by step into programming with Java. The necessary Java
techniques are explained in the context where they are required.
We start from the canonical ``Hello World'' program and end with an 
introduction to high performance parallel computing not only in Java.
Furthermore, the platform independence of Java and its free
availability are  a great advantage. Students can practice in the class 
and at home on different platforms without having to buy expensive
software. 

